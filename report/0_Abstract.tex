Most people believe that climate change is in some way happening. More importantly, many people are convinced that it will have tremendous consequences for the current generation as well as for those yet to come. Although it is common knowledge that changes are going on, there are just few that take action. As the human race is not self-destructive in its self, this inactivity might be explained by the Optimism Bias. In short, people feel affected by threats that seem more impending to them privately. This fosters the believe that catastrophic events like bushfires in Australia are being far away from them and therefore making them less concerned. We want to overcome that bias by utilizing Machine Learning. To do so, our work is inspired by the contributions of Viktor Schmidt et. al. who implemented a Cycle-Consistent Generative Adversarial Network to predict how a house might look like after being hit by a flood. Our eventual goal now is to be able to take pictures of a house as an input and then create an image of the same house burning. The data is self collected, and the trained model currently only aims to generate pictures of burning houses. Our work should help people to envision possible future scenarios for their home and hence could lead to a stronger incentive to actively fight climate change.  